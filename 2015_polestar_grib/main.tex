\documentclass[xcolor=svgnames]{beamer}
\usetheme[
    %%% options passed to the outer theme
    %    hidetitle,           % hide the (short) title in the sidebar
    %    hideauthor,          % hide the (short) author in the sidebar
    %    hideinstitute,       % hide the (short) institute in the bottom of the sidebar
    %    shownavsym,          % show the navigation symbols
         width=1.5cm,         % width of the sidebar (default is 2 cm)
    %    hideothersubsections,% hide all subsections but the subsections in the current section
    %    hideallsubsections,  % hide all subsections
         left                 % right of left position of sidebar (default is right)
    %%% options passed to the color theme
        lightheaderbg         % use a light header background
  ]{AAUsidebar}

% #### graphics and schemes
\usepackage{graphicx}
\graphicspath{{img/}}
\usepackage{tikz}
\usetikzlibrary{                          % TikZ libraries
                scopes,                   % .
                shapes,                   % .
                arrows,                   % .
                through,                  % .
                calc,                     % .
                intersections,            % .
                spy,                      % .
                matrix,                   % .
                chains,                   % .
                decorations.pathreplacing,% .
                decorations.pathmorphing, % .
                decorations.markings}     % .

\usepackage{pgfplots}                     % TikZ plots
\usepackage{pgfplotstable}                % TikZ tables from CSV
\pgfplotsset{compat=1.3}                  % activates \xilabel shift` for pgfplots
\usepackage{array}
\usepackage{listings}
\usepackage{ccicons}
\usepackage{tcolorbox}
\usepackage{listings}                     % code
\usepackage{adjustbox}                    % code
\usepackage{eurosym}                      % euro symbol
\usepackage{attrib}

% #### colors
\usepackage{xcolor}                       % common color names
\usepackage{colortbl}                     % common color names

% #### layouts
\usepackage{multicol}
\usepackage[textfont=footnotesize,bf]{caption}
\usepackage{subfig}

% #### math
\usepackage{siunitx}

% #### fonts
\usepackage[utf8]{inputenc}
\usepackage[english]{babel}
\usepackage[T1]{fontenc}
\usepackage{cmbright}
\usepackage{soul} %slanted text
\usepackage{hyperref}
\urlstyle{same}
\hypersetup{pdfauthor={Francesco de Virgilio},pdftitle={Handling GRIB files: when NOT to use GIS to manage geographic data}}

% #### tables
\usepackage{booktabs}			          % migliora la qualità delle tabelle
\usepackage{tabularx}			          % colonne a spaziatura fissa delle tabelle
\newcommand{\otoprule}                    % better top rule horizontal line
    {\midrule[\heavyrulewidth]}           % .


\begin{document}

\usebackgroundtemplate{%
    \includegraphics[width=\paperwidth,height=\paperheight]{img/back_first}}

    \begin{frame}[plain,noframenumbering]
        \begin{center}
            \color{white}
            \LARGE{Handling weather GRIB files}\\
            \normalsize{When \textit{not} to use GIS to manage geographic data}\\
            \vspace{40pt}
            Francesco de Virgilio\\
            \vspace{8pt}
            \scriptsize{Pole Star Space Applications}\\
            \scriptsize{London, 27 Jan 2015}
        \end{center}
    \end{frame}

\usebackgroundtemplate{%
    \includegraphics[width=\paperwidth,height=\paperheight]{img/back_normal}}

\section{GIS}

    \begin{frame}{GIS in Pole Star}
        \begin{center}
            \color{black}
            \begin{block}{Geographic Information System}
                Is a computer system designed to capture, store, manipulate, analyze, manage, and present all types of spatial or geographical data.
            \end{block}
        \end{center}
    \end{frame}

% * gis software in polestar

        \begin{frame}
        \end{frame}

   \subsection{Why GIS?}

        \begin{frame}
        \end{frame}
 
    \subsection{Impracticalities}

        \begin{frame}
        \end{frame}

\section{Weather}

    \subsection{PurpleFinder}
    % * old approach: PuFi

        \begin{frame}
        \end{frame}

    \subsection{NumPy}
    % * new approach: NumPy parsing + API

        \begin{frame}
        \end{frame}

    \subsection{PostGIS 2}
    % * future approach: PostGIS raster + API

        \begin{frame}
        \end{frame}

\section{Space: grid}

    \begin{frame}
    \end{frame}

\section{Time: slots}

    \begin{frame}
    \end{frame}

\section{Response structure}

    \begin{frame}
    \end{frame}

\end{document}
